%%%%%%%%%%%%%%%%%%%%%%%%%%%%%%%%%%%%%%%%%
% File name: ICERM2019.tex
% Poster name: TBD
% 
%
% Created by: Henry Kvinge and Manuchehr Aminian
% Based on a template by:
% Computational Physics and Biophysics Group, Jacobs University
% https://teamwork.jacobs-university.de:8443/confluence/display/CoPandBiG/LaTeX+Poster
% 
% Further modified by:
% Nathaniel Johnston (nathaniel@njohnston.ca)
%
% This template has been downloaded from:
% http://www.LaTeXTemplates.com
%
% License:
% CC BY-NC-SA 3.0 (http://creativecommons.org/licenses/by-nc-sa/3.0/)
%
%%%%%%%%%%%%%%%%%%%%%%%%%%%%%%%%%%%%%%%%%

%----------------------------------------------------------------------------------------
%	PACKAGES AND OTHER DOCUMENT CONFIGURATIONS
%----------------------------------------------------------------------------------------

\documentclass[final]{beamer}
\usepackage[scale=1.24]{beamerposter} % Use the beamerposter package for laying out the poster
\usepackage{multicol, xcolor, tikz, dsfont,mathabx,mathtools}
\usepackage{multicol}
\usepackage{graphicx}

\usetheme{confposter} % Use the confposter theme supplied with this template


\setbeamercolor{block title}{fg=ngreen,bg=white} % Colors of the block titles
\setbeamercolor{block body}{fg=black,bg=white} % Colors of the body of blocks
\setbeamercolor{block alerted title}{fg=white,bg=dblue!70} % Colors of the highlighted block titles
\setbeamercolor{block alerted body}{fg=black,bg=dblue!10} % Colors of the body of highlighted blocks
% Many more colors are available for use in beamerthemeconfposter.sty

%-----------------------------------------------------------
% Define the column widths and overall poster size
% To set effective sepwid, onecolwid and twocolwid values, first choose how many columns you want and how much separation you want between columns
% In this template, the separation width chosen is 0.024 of the paper width and a 4-column layout
% onecolwid should therefore be (1-(# of columns+1)*sepwid)/# of columns e.g. (1-(4+1)*0.024)/4 = 0.22
% Set twocolwid to be (2*onecolwid)+sepwid = 0.464
% Set threecolwid to be (3*onecolwid)+2*sepwid = 0.708

\newlength{\sepwid}
\newlength{\onecolwid}
\newlength{\middlecolwid}
\setlength{\paperwidth}{48in} % A0 width: 46.8in
\setlength{\paperheight}{36in} % A0 height: 33.1in
\setlength{\sepwid}{0.024\paperwidth} % Separation width (white space) between columns
\setlength{\onecolwid}{0.3413\paperwidth} % Width of one column
\setlength{\middlecolwid}{0.22\paperwidth} % Width of two columns
\setlength{\topmargin}{-0.5in} % Reduce the top margin size
%-----------------------------------------------------------

%\usepackage{graphicx}  % Required for including images

\usepackage{booktabs} % Top and bottom rules for tables

%------------------------------- Tikz Related  ------------------------------------------------------------------------------------------

\usetikzlibrary{decorations.pathreplacing,shapes}
\usepackage{tikz}


%picture of clockwise bubble
\newcommand{\ckpicture}{
\begin{tikzpicture}
\draw[ultra thick] (0,0) circle (3cm);
\node at (-3,0) {\arrowlines};
\draw[fill=black] (-2.4,1.8) circle (.2cm);
\node at (-3,2.9) {$k$};
\end{tikzpicture}}

%picture of counterclockwise bubble
\newcommand{\ctildekpicture}{
\begin{tikzpicture}
\draw[ultra thick] (0,0) circle (3cm);
\node[rotate = 180] at (-3,0) {\arrowlines};
\draw[fill=black] (-2.4,1.8) circle (.2cm);
\node at (-3,2.9) {$k$};
\end{tikzpicture}}

%----------------------------------------------------------------------------------------
%	Omit command 
%----------------------------------------------------------------------------------------

\newcommand{\omitt}[1]{}

%----------------------------------------------------------------------------------------
%	TITLE SECTION 
%----------------------------------------------------------------------------------------

\title{Clever title: Using convolutional neural networks and wavelet transforms to under immune responses in mice} % Poster title

\author{Henry Kvinge, Manuchehr Aminian, Michael Kirby} % Author(s)

\institute{Colorado State University (henry.kvinge@colostate.edu)} % Institution(s)

%----------------------------------------------------------------------------------------

\begin{document}

\addtobeamertemplate{block end}{}{\vspace*{2ex}} % White space under blocks
\addtobeamertemplate{block alerted end}{}{\vspace*{2ex}} % White space under highlighted (alert) blocks

\setlength{\belowcaptionskip}{2ex} % White space under figures
\setlength\belowdisplayshortskip{2ex} % White space under equations

\begin{frame}[t] % The whole poster is enclosed in one beamer frame

\begin{columns}[t] % The whole poster consists of three major columns, the second of which is split into two columns twice - the [t] option aligns each column's content to the top

\begin{column}{\sepwid}\end{column} % Empty spacer column

\begin{column}{\onecolwid} % The first column

%----------------------------------------------------------------------------------------
%	Objective
%----------------------------------------------------------------------------------------

\setbeamercolor{block alerted title}{fg=black,bg=ngreen} % Change the alert block title colors
\setbeamercolor{block alerted body}{fg=black,bg=white} % Change the alert block body colors
\begin{alertblock}{Objective}

\end{alertblock}

\vspace{-4mm}

%----------------------------------------------------------------------------------------
%	Khovanov's Heisenberg category
%----------------------------------------------------------------------------------------

\begin{block}{Background}


\end{block}

\end{column} % End of the first column

%--------------------------------------------------------------------------------------------
%Break between first and second column
%--------------------------------------------------------------------------------------------

\begin{column}{\sepwid}\end{column} % Empty spacer column

\begin{column}{\middlecolwid} % Begin second column

\begin{alertblock}{The process} 


\end{alertblock}

\vspace{-5mm}

\begin{block}



\end{block}
\end{column} % End of the second column

%----------------------------------------------------------------------------------------
%	Third column
%----------------------------------------------------------------------------------------

\begin{column}{\sepwid}\end{column} % Empty spacer column

\begin{column}{\onecolwid} % Begin the third column

%----------------------------------------------------------------------------------------
%	Finishing shifted symmetric functions
%----------------------------------------------------------------------------------------


%----------------------------------------------------------------------------------------
%	Transition and co-transition measure
%----------------------------------------------------------------------------------------

\begin{block}{Transition functions on the Schur graph}



 
\end{block}

\vspace{-8mm}


%----------------------------------------------------------------------------------------
%	Acknowledgements
%----------------------------------------------------------------------------------------

\setbeamercolor{block title}{fg=red,bg=white} % Change the block title color

\begin{block}{Acknowledgements}

\small{\rmfamily{We would like to thank...}} \\

\end{block}


%----------------------------------------------------------------------------------------
%	REFERENCES
%----------------------------------------------------------------------------------------

\begin{block}{References}

%\nocite{*} % Insert publications even if they are not cited in the poster
%\small{\bibliographystyle{unsrt}
\small{\bibliographystyle{amsplain}
\bibliography{spinHeisenCenter}\vspace{0.5in}}

\end{block}

\end{column} % End the third column

\end{columns} % End of all the columns in the poster

\end{frame} % End of the enclosing frame

\end{document}
